
\documentclass[a4paper,12pt,titlepage]{article}
\usepackage[utf8]{inputenc} 
\usepackage{tikz,pgf}
\usepackage{indentfirst}
\usepackage{amsfonts}
\usepackage[english]{babel}


%Url e Bookmarks of output PDF 
\usepackage{hyperref}
\hypersetup{
	colorlinks=true,
	linkcolor=blue,
	filecolor=magenta,      
	urlcolor=cyan,
	%	pdftitle={Document title},
	bookmarks=true,
	%pdfpagemode=FullScreen,
}




\usepackage{rotating}

\usepackage{tabularx}
\usepackage{multirow} 
\usepackage{lscape}
\usepackage{tikz}
%to insert PDF files
\usepackage[final]{pdfpages}
%--Packages--

\usepackage{eurosym}
\usepackage{graphicx} \usepackage{verbatim}
\usepackage{graphics}
\usepackage{tikz,pgf}
\usepackage{indentfirst}
\usepackage{amsfonts}
\usepackage{graphicx}
\usepackage{amsmath}
\usepackage{amsmath,amssymb,amsthm,textcomp}
\usepackage{enumerate}
\usepackage{multicol}
\usepackage{tikz}
\usepackage{geometry}
\usepackage{mathtools}
\usepackage{amsmath}
\usepackage{verbatim}
\usepackage{amsmath,amssymb,mathrsfs}
\usepackage{xcolor}
\usepackage{graphicx,color,listings}
\frenchspacing 
\usepackage{geometry}
\usepackage{rotating}
\usepackage{caption}
\usepackage{xcolor}
\usepackage{listings}
%Cool maths printing
\usepackage{amsmath}
%PseudoCode
\usepackage{algorithm2e}


\begin{document}
\section*{Two-Echelon Multicommodity Location Model}
\subsection*{Sets}
- $I$ = set of production plants; (\textit{index $i$})\\
- $J$ = set of potential Distribution Centers (DCs); (\textit{index $j$})\\
- $R$ = set of demand nodes; (\textit{index $r$})\\
- $K$ = set of homogeneous commodities; (\textit{index $k$})
\subsection*{Parameters}
- $p$ = maximum number of DCs that can be opened;\\
- $c_{ijr}^k$ = unit transportation cost of commodity $k\in K$ from plant node $i\in I$ to demand node $r\in R$ across DC $j\in J$;\\
- $d_{r}^k$ = demand of commodity $k\in K$ from demand node $r\in R$;\\
- $p_{i}^k$ = maximum quantity of commodity $k\in K$ that can be manufactured by plant $i\in I$;\\
- $q_{j}^-$ = minimum activity level of potential DC $j\in J$;\\
- $q_{j}^+$ = maximum activity level of potential DC $j\in J$;\\
- $f_{j}$ = fixed cost of potential DC $j\in J$;\\
- $g_{j}$ = marginal cost of potential DC $j\in J$;
\subsection*{Variables}

$$z_j=
\bigg \{
\begin{array}{ll}
1\qquad \text{if DC $j\in J$ is opened}\\
0\qquad \text{otherwise}\\
\end{array}
$$

$$y_{jr}=
\bigg \{
\begin{array}{ll}
1\qquad \text{if demand node $r\in R$ is assigned to DC $j\in J$}\\
0\qquad \text{otherwise}\\
\end{array}
$$

- $s_{ijr}^k$ = amount of commodity $k\in K$ transported from plant node $i\in I$ to demand node $r\in R$ across DC $j\in J$;\\
\\
The following feasibility condition must hold:\\
\begin{equation*}
\sum_{i\in I}p_i^k \geq \sum_{r\in R} d_r^k  \quad      k\in K
\end{equation*}

%\newpage
\subsection*{TEMC Mathematical Formulation} 
\begin{equation*}
\sum_{i\in I}\sum_{j\in J}\sum_{r\in R}\sum_{k\in K} c_{ijr}^k \cdot s_{ijr}^k +
\sum_{j\in J} \left( f_j \cdot z_j + g_j \cdot \sum_{r\in R}\sum_{k\in K} d_r^k \cdot y_{jr}             \right) 
\end{equation*}

\begin{equation*}
\sum_{j\in J}\sum_{r\in R} s_{ijr}^k \leq p_i^k \quad i\in I,\,k\in K
\tag{1}
\end{equation*}

\begin{equation*}
\sum_{i\in I} s_{ijr}^k = d_r^k \cdot y_{jr} \quad j\in J,\,r\in R,\,k\in K
\tag{2}
\end{equation*}

\begin{equation*}
\sum_{j\in J}y_{jr} = 1 \quad r\in R
\tag{3}
\end{equation*}

\begin{equation*}
q_j^-\cdot z_j \leq \sum_{r\in R}\sum_{k\in K}d_r^k \cdot y_{jr} \leq q_j^+\cdot z_j \quad j\in J
\tag{4}
\end{equation*}

\begin{equation*}
\sum_{j\in J} z_j = p
\tag{5}
\end{equation*}

\begin{equation*}
z_j \in \left\lbrace0,1\right\rbrace \quad j\in J
\tag{6}
\end{equation*}

\begin{equation*}
y_{jr} \in \left\lbrace0,1\right\rbrace \quad j\in J,\, r\in R
\tag{7}
\end{equation*}

\begin{equation*}
s_{ijr}^k\geq 0 \quad i\in I,\,j\in J,\, r\in R,\,k\in K
\tag{8}
\end{equation*}



\subsection*{Demand Allocation Problem}
If a set $\overline{z}$ $j\in J$ and $\overline{y}_{jr}$, $j\in J,\,r\in R$ of feasible values is available, you just have to solve the following LP problem in order to determine the optimal demand allocation:
\begin{equation*}
\sum_{i\in I}\sum_{j\in J}\sum_{r\in R}\sum_{k\in K} c_{ijr}^k \cdot s_{ijr}^k 
\end{equation*}

\begin{equation*}
\sum_{j\in J}\sum_{r\in R} s_{ijr}^k  \leq p_i^k \quad i\in I,\,k\in K
\end{equation*}

\begin{equation*}
\sum_{i\in I} s_{ijr}^k = d_r^k \cdot \overline{y}_{jr} \quad j\in J,\,r\in R,\,k\in K
\end{equation*}

\begin{equation*}
s_{ijr}^k\geq 0 \quad i\in I,\,j\in J,\, r\in R,\,k\in K
\end{equation*}





\end{document}